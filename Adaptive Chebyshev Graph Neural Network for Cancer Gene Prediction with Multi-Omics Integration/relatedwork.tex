Graph Neural Networks (GNNs) have become indispensable in biological network analysis. In \cite{chatzianastasis2023emgnn}, the authors propose EMGNN, a multilayer graph neural network that integrates multiple gene-gene interaction networks with pan-cancer multi-omics data to improve cancer gene prediction. Unlike single-network approaches, EMGNN captures tumorigenesis complexity across diverse networks while incorporating interpretability features to enhance transparency.
Li and Nabavi~\cite{li2024multimodalgnn} introduce a multimodal GNN for cancer molecular subtype classification, leveraging graph-based representations rather than traditional early or late fusion methods. Similarly, Li et al.~\cite{li2023cgmega} develop CGMega, an explainable graph neural network with attention mechanisms for cancer gene module dissection.
Peng et al.\cite{peng2022cancerdriver} present MTGCN, a Multi-Task Graph Convolutional Network (GCN) that integrates PPI networks for improved driver gene identification. Further, Peng et al.\cite{peng2024multinetwork} propose MNGCL, a contrastive learning framework integrating multi-omics datasets to enhance information interaction.

Schulte-Sasse et al.\cite{schulte2021multiomics} develop EMOGI, a GCN-based explainable method integrating pan-cancer multi-omics and PPI networks, using layer-wise relevance propagation for interpretability. Zhang et al.\cite{zhang2023heterophilic} introduce HGDC, a Heterophilic Graph Diffusion Convolutional Network, utilizing Personalized PageRank (PPR) for improved message propagation in heterogeneous networks.
Zhao et al.~\cite{zhao2022modig} propose MODIG, a GAT-based framework integrating multi-omics data with diverse gene networks, including co-expression patterns and KEGG pathways, to enhance driver gene identification.

Topology Adaptive Graph Convolutional Networks (TAGCNs)\cite{du2018topology} extend traditional GCNs by dynamically adjusting convolutional operations based on graph structure, capturing both local and global topology. Singh et al.\cite{singh2024gtagcn} further enhance this with GTAGCN, combining Generalized Aggregation Networks with TAGCNs for versatile applications. Despite these advancements, the potential of pretrained embeddings in multi-omics data integration remains underexplored.

Our methode optimizes the Chebyshev polynomial order, eliminates redundant computations, and employs efficient normalization techniques to enhance scalability without compromising predictive accuracy. While the Chebyshev Graph Convolutional Network (ChebNet) is a powerful spectral GNN variant, its high computational cost poses challenges. ACGNN addresses these limitations by improving computational efficiency while preserving interpretability and accuracy, making it well-suited for large-scale biological applications.
